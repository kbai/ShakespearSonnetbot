%
\subsection{Rhyme}\label{sec:rhymedict}
The naive approach to generate poems does not honor the rhyme pattern in the Shakespear's sonnets. However, it is actually not difficult to introduce rhyme in our group-based poem generating algorithms. There are two steps to generate poems with rhymes: first, build a rhyming dictionary; second, seed the end of the line with words that rhyme, and then do HMM generation in the reverse direction.

To build a rhyming dictionary, we pick out the last words of pair of rhyming lines. If these two words rhyme, we add it to the rhyming dictionary. For example, ``increase" and ``decease" rhyme with the same phonetic ``IY-S" (\texttt{cmudict} phonetic form), and thus we generate an phonetic item ``IY-S" in the rhyming dictionary with two words ``increase" and ``decease". In Sonnet 11, we find another pair ``increase" and ``cease" also rhyme with ``IY-S". Then, we add ``cease" into the item ``IY-S". Sometimes, the last words of pair of rhyming lines do not rhyme, like ``die" and ``memory". In this case, we only check these two words separately whether they can be added into some existing phonetic item. For example, ``die" can be added into the phonetic item ``AY" and ``memory" can be added into the phonetic item ``IY". After traversing all the rhyming lines twice, we build a rhyming dictionary for each corpus. 

We combine the \textit{NLTK} package and the RhymeBrain website~\url{http://rhymebrain.com/en} to identify whether two words rhyme or not and the phonetic they rhyme. For the word which exists in \texttt{cmudict} (in \textit{NLTK}), we use \textit{NLTK} to get its phonetics. For example, 
\begin{lstlisting}
>>> phondict = nltk.corpus.cmudict.dict()
>>> phondict['increase']
[[u'IH0', u'N', u'K', u'R', u'IY1', u'S'], [u'IH1', u'N', u'K', u'R', u'IY2', u'S']]
\end{lstlisting}
If either of the pronunciation rhymes with other word, like ``cease", we think that these two words rhyme. For the word which does not exists in \texttt{cmudict}, our script \textit{automatically} picks an auxiliary word which rhyme with this word from the RhymeBrain website~\url{http://rhymebrain.com/en}, and use the phonetics of the auxiliary word to analyze the rhyme. For example, ``fulfil" does not exists in \texttt{cmudict}. Our script will go to the RhymeBrain website and pick the word ``foothill". Then we use 
\begin{lstlisting}
>>> phondict['foothill']
[[u'F', u'UH1', u'T', u'HH', u'IH2', u'L']]
\end{lstlisting}
to analyze whether ``fulfil" and ``will" rhyme, and the answer is yes!

In this case, we train the HMM or 2rd-order Markov model in the reverse direction. To do this, we only need to reverse every line in the input corpus. To generate a poem, we first seed the end of the line with words that rhyme, and then generate lines with the reverse-direction-trained model. \textit{The following are several rhyming lines generated by trained HMM for \texttt{groupG} with number of hidden state 80.}
\settowidth{\versewidth}{even  see  shall  accessary  used  must  find  and  herself  enfeebled  mine  it}
\begin{verse}[\versewidth]
 as  with  proves  replete  in  thee  writ \\
 even  see  shall  accessary  used  must  find  and  herself  enfeebled  mine  it  \\
 she  this  and  thee  praise  \\
 then  love  away  night  seat  is  one  days  \\
 this  even  had  in  lived  their  young  part  \\
 yet  subjects  knife  what  right  winter  thee  heart  \\
 and  thou  feelst  find  bear  wretchcd  your  store  line  \\
 that  other  not  may  it  of  shows  this  mine  in  writ  mine  \\
 pity  mayst  be  you  made  to  praise  best  \\
 the  thee  be  him  and  length  time  thou  am  still  breast  
\end{verse}
We can see that the lines alway rhyme, but the number of syllables varies a lot.

\subsection{Controlling the total number of syllables in a line}\label{sec:syllablecount}
There are several ways to control the total number of syllables in each line and ideally to make it exactly 10. We take a very simple approach to do this: repeatedly generating lines until the total number of syllables is 10. Sometimes, it takes a long time to get a line with exactly 10 syllables. Therefore, we randomly generate at most 50 lines, and keep the line whose total number of syllables is closest to 10.

To count the total number of syllables in each line, we need to count the number of syllables in each word. We combine the \textit{NLTK} package, the \textit{PyHyphen} package and our own-written function \texttt{count\_syllables()} to count the number of syllables in each word as accurate as possible. If a word is in \texttt{cmudict}, the \textit{NLTK} gives us the right answer. For examples, ``increase" has 2 syllables according to \texttt{cmudict}. If a word is not in \texttt{cmudict}, we use \textit{PyHyphen} to count the syllables. For example,
\begin{lstlisting}
In[4]: from hyphen import Hyphenator
In[5]:  h_en = Hyphenator('en_US')
In[6]: len(h_en.syllables(unicode('fulfil')))
Out[6]: 2
In[7]: len(h_en.syllables(unicode('air')))
Out[7]: 0
\end{lstlisting}
We can see that the \textit{PyHyphen} package is not so accurate to identify the number of syllables in a word. Therefore, we also write our own function \texttt{count\_syllables()} (see file \texttt{countvowel.py}) to correct possible mistakes made by the \textit{PyHyphen} package.

With this approach to control the total number of syllables in a line, we get rhyming lines all of which have total number of syllables nearly 10. \textit{The following are several rhyming lines generated by trained 2rd-order Markov model for \texttt{groupG}.}
\settowidth{\versewidth}{even  see  shall  accessary  used  must  find  and  herself  enfeebled  mine  it}
\begin{verse}[\versewidth]
 which  die  for  goodness  who  have  lived  for  crime \\
 but  were  some  child  of  yours  alive  that  time \\
 to  give  back  again  and  straight  grow  sad \\
 this  told  joy  but  then  no  longer  glad \\
 lo  thus  by  day  my  limbs  by  night  my  mind \\
 for  thee  and  for  my  name  thy  love  and  am  blind \\
 think  all  but  one  and  me  most  wretchcd  make \\
 till  then  not  show  my  head  where  thou  mayst  take \\
 so  till  the  judgment  that  your  self  arise \\
 so  long  lives  this  and  dwell  in  lovers  eyes
\end{verse}
We can see that the lines alway rhyme, and the total number of syllables in each line is nearly 10.


\subsection{Incorporating additional texts} \label{sec:additionaltext}
Our framework enables us to train our models with additional texts. We include all 139 of Spenser's sonnets in our training datasets. With the same process, i.e., pre-processing, rhyme dictionary learning, model training (for both HMM and 2rd-order Markov model), we can easily get models which have a larger dictionary. The training time nearly get doubled because we have nearly double sized training data. The following is one poem from our trained 2rd-order Markov model, with rhyming and controlling-the-total-number-of-syllables.
\settowidth{\versewidth}{even  see  shall  accessary  used  must  find  and  herself  enfeebled  mine  it}
\begin{verse}[\versewidth]
 the  dedicated  words  which  writers  use \\
 to  new  found  methods  and  to  compounds  strange \\
 as  fast  as  thou  art  too  dear  for  my  muse \\
 so  far  from  variation  or  quick  change \\
 reserve  them  for  my  love  doth  well  denote \\
 or  from  their  proud  lap  pluck  them  where  they  grew \\
 if  that  be  fair  whereon  my  false  eyes  dote \\
 at  wondrous  sight  of  so  celestial  hew \\
 prison  my  heart  with  silence  secretly \\
 and  sweets  grown  common  lose  their  dear  delight \\
 but  rising  at  thy  name  doth  point  out  thee \\
 than  when  her  mournful  hymns  did  hush  the  night \\
 \vin so  return  rebuked  to  my  content \\
 \vin that  it  hereafter  may  you  not  repent
\end{verse}

After examination, we find that the first line ``the  dedicated  words  which  writers  use" is actually borrowed from Shakespear sonnet 82, and the eighth line ``at  wondrous  sight  of  so  celestial  hew" is borrowed from Spenser sonnet 3. Other lines are not directly borrowed from either of them, but are kind of mixture of them. For example, the third line "as  fast  as  thou  art  too  dear  for  my  muse" is a combination of  ``As fast as thou shalt wane so fast thou grow'st" from Shakespere sonnet 11 and ``So oft have I invoked thee for my muse" from Shakespere sonnet 78. By introducing additional texts (spenser's poems), we get more variations in the poems we generate.

\subsection{Adding punctuations}\label{sec:punctuations}


