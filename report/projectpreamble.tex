\documentclass{article}
\usepackage{amsmath}
%\usepackage{subfigure}
\usepackage{subfig}
\usepackage{wrapfig}
\usepackage{amsthm}
\usepackage{amssymb}
\usepackage{graphicx}
\usepackage{mdwlist}
\usepackage[colorlinks=true]{hyperref}
\usepackage{geometry}
\usepackage{titlesec}
\geometry{margin=1in}
\geometry{headheight=2in}
\geometry{top=2in}
\usepackage{palatino}
% \usepackage{mathrsfs}
\usepackage{fancyhdr}
\usepackage{paralist}
% \usepackage{todonotes}
\setlength{\marginparwidth}{2.15cm}
\usepackage{tikz}
\usetikzlibrary{positioning,shapes,backgrounds}
\usepackage{float} % Place figures where you ACTUALLY want it
\usepackage{comment}
\usepackage{ifthen}
\rhead{}
\lhead{}

\renewcommand{\baselinestretch}{1.15}

% Shortcuts for commonly used operators
\newcommand{\E}{\mathbb{E}}
\newcommand{\Var}{\operatorname{Var}}
\newcommand{\Cov}{\operatorname{Cov}}
\DeclareMathOperator{\argmin}{arg\,min}
\DeclareMathOperator{\argmax}{arg\,max}

% do not number subsection and below
\setcounter{secnumdepth}{1}

% custom format subsection
\titleformat*{\subsection}{\large\bfseries}

% set up the \question shortcut
\newcounter{question}[section]
\newenvironment{question}[1][]
  {\refstepcounter{question}\par\addvspace{1em}\textbf{Question~\Alph{question}\!
    \ifthenelse{\equal{#1}{}}{}{ [#1 points]}: }}
    {\par\vspace{\baselineskip}}

\newcounter{subquestion}[question]
\newenvironment{subquestion}[1][]
  {\refstepcounter{subquestion}\par\medskip\textbf{\roman{subquestion}.\!
    \ifthenelse{\equal{#1}{}}{}{ [#1 points]:}} }
  {\par\addvspace{\baselineskip}}

\titlespacing\section{0pt}{12pt plus 2pt minus 2pt}{0pt plus 2pt minus 2pt}
\titlespacing\subsection{0pt}{12pt plus 4pt minus 2pt}{0pt plus 2pt minus 2pt}
\titlespacing\subsubsection{0pt}{12pt plus 4pt minus 2pt}{0pt plus 2pt minus 2pt}
